	\pagestyle{plain}
	\setcounter{page}{1}
	\pagenumbering{arabic}
	\newpage
	
	{\large {\textbf{ELEN4009 SOFTWARE ENGINEERING PROJECT\\}Fast Kung 604087 \& David Golach 675717 \& Jonathan Atkins 675213}
	}
	\section{DESCRIPTION}

	The business of real estate is a two sided affair, one side being the seller of the house and another being the one of the home seeker. To be of benefit to both parties to sell or find the house of interest, an application can be developed. This project thus seeks to develop an intelligent ideal home locator. This application will provide a home seeker with an ideal location of residence based on certain preferences that they must provide. These preferences include factors such as crime rate of the area to the nearest shopping districts or restaurants. House sellers in these locations will also benefit from the application through the advertising their area will receive. This project will be split into a front and back end, creating a user interface and database accessed by a weighted priority search algorithm respectively.
	 
	\section{DEVELOPMENT PROCESS}
	
	\subsection{Front-end task}
	
	Front-end tasks require the creation of a user interface that allows for input preferences. These inputs will be entered into the application through the following types:
	
	\begin{itemize}
		\item Preferred area (from a drop down box) 
		\item Work location (from a drop down box)
		\item All other preferences (Priority slider)
		\item Number of locations (number slider)
		\item Possibly current location to find ideal nearby areas of residence
	\end{itemize}
	
	 The user interface will take in these inputs and then reveal the ideal locations of residence. Each area will also have their statistics displayed. Possibly, recommendations on school choices, quality of restaurants or any other relevant information can also be provided. Pictures or maps of areas can also be provided if possible.
	 
	
	\subsection{Back-end task}
	
	 Back-end tasks require creation of a priority weighted search algorithm. This will be used to calculate the ideal locations from the input data received. These areas will be found according to a database of locations that will be compiled. This database will include, but is not limited to:

	\begin{itemize}
		\item Crime Rate
		\item Shopping Districts
		\item Restaurants
		\item Neighbourhood Impact
		\item Air Pollution
		\item Schools
		\item Work Location 
		\item Travel and Traffic
		\item Price
		\item Public Transport
	\end{itemize}	 
	
	\subsection{General Task Specifications}
	
	The application will be created using the G++ coding language, with the user interface utilizing a GUI such as SFML. Any new development will be documented in further progress reports. These documents, and all coding done, will be uploaded to a public GITHUB repository for access and updates. The name of the repository and link are given below:
	
		\textbf{LINK:} https://github.com/675213/ELEN4009LabRepo.git
		\\
		\textbf{NAME:} ELEN4009LabRepo
	\section{RESPONSIBILITIES}
	
	A group of three students will be developing this application. Each will actively participate in the entire development process, but each student will focus on a specific part of the project, namely:
	
	\begin{itemize}
	\item Student 1: Compilation of database
	\item Student 2: Algorithm and back end programming
	\item Student 3: Front end programming and documentation
	\end{itemize}

	\section{CONCLUSION}

	This is not a completed document, as per the requirements for this initial exercise. However, it can be concluded that this application is feasible and development can begin.






